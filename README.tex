% Created 2010-08-07 Sab 15:27
\documentclass[11pt]{article}
\usepackage[utf8]{inputenc}
\usepackage[T1]{fontenc}
\usepackage{fixltx2e}
\usepackage{graphicx}
\usepackage{longtable}
\usepackage{float}
\usepackage{wrapfig}
\usepackage{soul}
\usepackage{t1enc}
\usepackage{textcomp}
\usepackage{marvosym}
\usepackage{wasysym}
\usepackage{latexsym}
\usepackage{amssymb}
\usepackage{hyperref}
\tolerance=1000
\providecommand{\alert}[1]{\textbf{#1}}

\title{MotNet}
\author{Andrea Crotti, Marius Grysla, Oscar Dustmann}
\date{07 Agosto 2010}

\begin{document}

\maketitle

\setcounter{tocdepth}{3}
\tableofcontents
\vspace*{1cm}

\section{Next Improvements}
\label{sec-1}
\subsection{\textbf{DONE} add support for adaptative compression}
\label{sec-1_1}

   \texttt{CLOSED:} \textit{2010-07-31 Sab 18:21}

\begin{itemize}
\item CLOSING NOTE \textit{2010-07-31 Sab 18:21}
\end{itemize}
\subsection{\textbf{DONE} add cleaning exit code on exit}
\label{sec-1_2}

   \texttt{CLOSED:} \textit{2010-08-02 Lun 11:21}

\begin{itemize}
\item CLOSING NOTE \textit{2010-08-02 Lun 11:21} \\
Using signal and a generic closing function!
\end{itemize}

     
\subsection{\textbf{TODO} add a simple handshaking protocol removing the ugly forced sleep}
\label{sec-1_3}
\subsection{\textbf{TODO} add support for an intermediate node that forwards everything}
\label{sec-1_4}
\subsection{\textbf{TODO} remove code duplication from client/gateway source files}
\label{sec-1_5}
\subsection{\textbf{TODO} (optional) some support for multiple clients}
\label{sec-1_6}
\subsection{\textbf{TODO} uniform code style (\_{} and doc)}
\label{sec-1_7}
\subsection{\textbf{TODO} do some real automated tests to compute real speed of the connection}
\label{sec-1_8}
\subsection{\textbf{TODO} remove all the (void)var things, if there is a variable use it ;)}
\label{sec-1_9}
\subsection{\textbf{TODO} drop privileges when not needed anymore}
\label{sec-1_10}

   ``setgid(),initgroups(), then setuid()'' should be more or less our system calls.
   Drop to another less privileged user must be done correctly.
   We also might fork the program keeping the tunnel manager always on
\subsection{\textbf{TODO} testing with a pipe instead of using the real usb device}
\label{sec-1_11}

\begin{verbatim}
pid_t pid;
int mypipe[2];

/* Create the pipe.  */
if (pipe (mypipe))
    {
        fprintf (stderr, "Pipe failed.\n");
        return EXIT_FAILURE;
    }
\end{verbatim}
    If is enought to simulate a serial interface we can use  \href{http://www.kernel.org/doc/man-pages/online/pages/man7/pty.7.html}{pty usage}
    Probably is enough to simulate a pty with a master/slave thing.

    
\subsection{\textbf{TODO} Finally fix the stupid tossim simulation to see how it works}
\label{sec-1_12}
\section{Goal}
\label{sec-2}

  The goal of this project is to create a create a network between motes.
  Supposing we have mote A attached to a computer with an internet connection and mote b is attached to a computer without internet connection.

  Then traversing a theoretically arbitrary network of motes we can let B connect to the internet through the network.
\section{Architecture}
\label{sec-3}

  
\section{Coding style}
\label{sec-4}
\section{Files}
\label{sec-5}

  This is the tree of files in our application
\begin{itemize}
\item driver
    In this directory we have client and gateway program, written in C for linux systems with some bash scripts.

\begin{itemize}
\item reconstruct.c
      this module in charge of reconstructing the chunks we get from from the network
\item chunker.c
      functions to split the message into many chunks
\item client.c
      start the client version of the program
\item gateway.c
      start the gateway
\item tunnel.c
      manage the tunnel (open close write read)
\item setup.c
      all the functions used both by the client and the gateway
\item structs.c
      contain some useful functions to manage our own data structures
\item motecomm.c
      low level communication between motes and driver program
\item glue.c
\end{itemize}

\item shared
    In this directory we keep the data structures definition that we use both from the client/gateway program and the program installed on the motes

\begin{itemize}
\item structs.h
      This contains the basic data structures we use communicate with the motes.
      They are shared between the driver program and program that runs on the motes
\end{itemize}

\item motes-simple

\begin{itemize}
\item SimpleMoteAppC.nc
      configuration file
\end{itemize}

\end{itemize}

\end{document}
