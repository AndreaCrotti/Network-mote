% Created 2010-08-17 Mar 18:32
\documentclass[11pt]{article}
\usepackage[utf8]{inputenc}
\usepackage[T1]{fontenc}
\usepackage{fixltx2e}
\usepackage{graphicx}
\usepackage{longtable}
\usepackage{float}
\usepackage{wrapfig}
\usepackage{soul}
\usepackage{t1enc}
\usepackage{textcomp}
\usepackage{marvosym}
\usepackage{wasysym}
\usepackage{latexsym}
\usepackage{amssymb}
\usepackage{hyperref}
\tolerance=1000
\providecommand{\alert}[1]{\textbf{#1}}

\title{MOTNET PROJECT}
\author{Andrea Crotti, Marius Grysla, Oscar Dustmann}
\date{17 Agosto 2010}

\begin{document}

\maketitle


\section*{Goal}
\label{sec-1}

  The goal of this project is to create a create a network between motes.
  Supposing we have mote A attached to a computer with an internet connection and mote b is attached to a computer without internet connection.

  Then traversing a theoretically arbitrary network of motes we can let B connect to the internet through the network.
\section*{How to use it}
\label{sec-2}

\begin{itemize}
\item Attach the motes on two computers \emph{A}, \emph{B}, where \emph{A} has internet access while \emph{B} doesn't
\item login as root (needed for tun device manipulation)
\item check that the motes are connected (motelist)
\item setup the tinyos environment

\begin{itemize}
\item setup TOSROOT and other useful variables, use \href{http://www.5secondfuse.com/tinyos/.bash_tinyos}{.bash\_tinyos} to do it automatically
\item follow the \href{http://docs.tinyos.net/index.php/BLIP_Tutorial}{blip tutorial} to compile the serialforwarder
\end{itemize}

\item install the program on both motes with:

\begin{itemize}
\item \textbf{cd motes-simple \&\& make telosb install /dev/ttyUSB0,1}
\end{itemize}

\item compile the driver with

\begin{itemize}
\item \textbf{make force}
\end{itemize}

\item on \emph{A} run the gateway with \textbf{./gateway /dev/ttyUSB0 wlan0} (for example)
\item on \emph{B} run the client with \textbf{make run}
\end{itemize}

  
  Now B should also have internet access, try to ping outside to check if it works.
\section*{Further documentation}
\label{sec-3}

  In \textbf{driver} you can run \textbf{make doc} to generate the doxygen documentation of the code, which you will find in \textbf{doc$_{\mathrm{doxy}}$}
\section*{Files}
\label{sec-4}

  This is the tree of files in our application
\begin{itemize}
\item \textbf{driver}
    In this directory we have client and gateway program, written in C for linux systems with some bash scripts.

\begin{itemize}
\item \textbf{reconstruct.c}
      this module in charge of reconstructing the chunks we get from from the network
\item \textbf{chunker.c}
      functions to split the message into many chunks
\item \textbf{client.c}
      start the client version of the program
\item \textbf{gateway.c}
      start the gateway
\item \textbf{tunnel.c}
      manage the tunnel (open close write read)
\item \textbf{setup.c}
      all the functions used both by the client and the gateway
\item \textbf{structs.c}
      contain some useful functions to manage our own data structures
\item \textbf{motecomm.c}
      low level communication between motes and driver program
\item \textbf{glue.c}
      Wrapper for the select system calls, glues several file descriptors together
\item \textbf{serialforwardif.c}
      Serial implementation using the serial forwarder for the pc side
\item \textbf{util.c}
      constructor for classes like objects
\end{itemize}

\item \textbf{motes-simple}

\begin{itemize}
\item \textbf{SimpleMoteAppC.nc}
      configuration file
\end{itemize}

\item \textbf{motes}
    Infrastructure for a more generic mote program supporting blip, not compatible with actual driver.
\item \textbf{python}
    Contains experimental python code creating structs, testing packets chunking and compression
\item \textbf{talks}

\begin{itemize}
\item \textbf{slides.org}
      org-mode source file of the presentation
\item \textbf{slides.tex}
      tex beamer generated file from slides.org
\item \textbf{.svg}
      images used
\item \textbf{slides.pdf}
      resulting slides
\end{itemize}

\end{itemize}

\end{document}
